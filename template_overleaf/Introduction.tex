%%%%%%%%%%%%%%%%%%%%%%%%%%%%%%%%%%%%%%%%%%%%%%%%%%%%%%%%%%%%%%%%%%%%%%%%%%
% Information about Apteronotus Leptorhynchus 
%%%%%%%%%%%%%%%%%%%%%%%%%%%%%%%%%%%%%%%%%%%%%%%%%%%%%%%%%%%%%%%%%%%%%%%%%%

\chapter{Introduction} 
    \label{Introduction}
    
We, as human beings, hear sounds – the melody of a song, a rustle or a bang. We smell the aroma of coffee, the scent of cinnamon or the stench of foulness. We see the shape of an animal or the bright colour of the sun. We taste a fruit’s sweetness or the bitterness of drugs. We feel the rough surface of sandpaper or the heat of fire. But how does it feel, if there are irritations in the electric field around us? One cannot imagine how other types of perception feel, how accurate they are in terms of object detection, time or space. 

 Yet, other species use types of perception successfully that are unfamiliar to us. The most popular examples are probably bats and some whales that use echolocation to detect and locate objects. However, there are some less prominent examples as well, one of which is the central subject of this thesis: active electroception. It is present in different species but will be analysed in \textit{Apteronotus leptorhynchus}, a weak electric fish of the South American rivers. \textit{A. leptorhynchus}, was first recorded by Ellis in \citeA{eigenmann1905sternarchorhamphus} as \textit{Sternarchus leptorhynchus} and belongs to the order of Gymnotiformes. They share the electric sense only with the family of Mormyriformes, which are native to the rivers of Africa. Their most common ancestor did not possess an active electric sense and  the ancestors in between did neither. This separate evolution of the same system, called analogy, is what most likely happened for active electroception in Gymnotiformes and Mormyriformes. However, what drove active electroception to be advantegous for survival, remains unclear.
 
 To be able to understand this advantage, it is first necessary to understand its basic characteristics. Active electroception or electroreception results from weak electric organ discharges (EODs) released by the fish's electric organ, as the name already indicates. The description of the electroception as being active in this case refers to the self-generation of the EODs. 
To perceive the electric field that has been modified by the fish's surrounding area, there are tuberous electroreceptors spread through the fish's epidermis. They are specific to amplitude and phase of the electric signal \cite{assad1997electric}.

The signals perceived by the electroreceptors are thought to be used for two different functions: electrocommunication and electrolocation \cite{binder2009encyclopedia}. Electrocommunication is defined as being an information transfer between two electric fish based on their EODs. This includes, but is not limited to: Age, sex, individual identity and motivational state \cite{binder2009encyclopedia}. The presence and functionality of this communicative use of electroreception are relatively well studied \cite<e.g.>{fugere2010electric, bastian2001arginine, hupe2008electrocommunication} and seems to be used commonly by all weakly electric fish. In contrast, the presence and importance of electrolocation in \textit{A. leptorhynchus} is still under debate. Electrolocation, on the other hand, describes the detection of objects using distortions in the electric field and the resulting electroreceptors' responses \cite{binder2009encyclopedia}. Objects causing a reduction of the electric current flow are those with an impedance value higher than the water. The opposite effect, an increase of the electric current flow, results from objects with low impedance values \cite{bullock2006electroreception}. With that, it should be theoretically possible for the fish to receive some information about objects close-by.

Even if one assumes that the latter function is present as well, it is not obvious which advantage it brings to the animal processing the information. In which situations could electrolocation possibly be more informative than the visual information received? It has been suggested that the lack of sufficient light as present in turbid waters may lead to difficulties with the visual system. Therefore, electrolocation may be a useful replacement system to gain the necessary information \cite{assad1997electric}, although this hypothesis only holds if the species under research does live in areas with a lack of light. This is not the case for \textit{A. leptorhynchus} as they inhabitate mainly rivers with clear water that enables them to use their visual system appropriately. Other explanations have not yet been presented. Therefore, the advantage that may result from the evolution of electrolocation is not easily explained. One characteristic that needs to be taken into account though, is the nocturnal activity that \textit{A. leptorhynchus} show \cite{raab2019dominance}. The same is probably true for other species of Gymnotiformes \cite{kramer2009encyclopedia} and with that, there could be a connection in terms of evolution between the presence of the electric sense and the fish being nocturnal. 

To understand the processes underlying electrolocation it is necessary to conceive the evolutionary process that produced it. In particular, it contributes to differentiate between being an analogous development or just a by-product. That is the aim of the project, this thesis is part of: building an in silico model of \textit{A. leptorhynchus}. The model is meant to include the fish itself, its generated electric field and its electroreceptor's responses. The first part, the implementation of the fish's basic anatomy has been the goal of this thesis. The majority of models developed so far use very simple geometrical models of the fish's contour to approximate the acquired information \cite[among others]{bacher1983new, chen2005modeling}. \citeA{fujita2010modeling} modeled the fish's body shape even with a rectangle and both other mentioned papers did not put much attention to the shape. The study of \citeA{babineau2006modeling} has indicated, though, that the fish's body shape is essential for getting informative electroreceptors' responses. The authors compare two simple geometrical models to one describing \textit{A. leptorhynchus} contour more accurately. The latter one can account for the presence of most electroreceptors in the rostral region - the model leads to a smooth uniform electric field at exactly this location. However, Babineau et al. (2006) did not take into account the exact body contour of the fish. This is a necessary precondition to be able to model the electric field and the corresponding responses accurately. 

Following this line of research, we assumed that acquiring exact anatomical data of \textit{A. leptorhynchus} to base a model on, is the most reliable way to generate accurate data later on. Hence, we acquired an MRI scan of a recently expired specimen. The resulting sections through the fish's body were then used for the geometrical approximation of the body contour. The whole geometrical model is based on the backbone and is constructed relative to its position. Thus, the backbone is introduced first by a definition of a hyperbola that allows us to model the bending of the fish's tail. Secondly, the according Frenet coordinate frame at any point on the backbone was constructed to then place the cross-section within the according normal plane. The form of the cross-sections themselves was directly based on the MRI paratransverse sections. A parameterized ellipsis was edited such that a tip at its ventral side is possible to approximate the sections more accurately. Adding the information taken from the MRI data to the edited ellipses was done by applying a simplex optimization over 60 paratransverse sections. To get continuous data at any point on the backbone, a polynomial regression over the three parameters the ellipses were based on has been executed. With that, it is possible to generate the outer contour of the fish at any point on its backbone. The implementation of these steps leads to a function classifying a point given in space as being either on the inside or on the outside the fish. Furthermore, some inner parts of the fish were considered as well: its spinal chord, its backbone and its electric organ. The position of these parts was approximated relative to the backbone's center as well and points localized within one of them not only get the label 'inside' but 'spinal chord', 'backbone' or 'electric organ'. 