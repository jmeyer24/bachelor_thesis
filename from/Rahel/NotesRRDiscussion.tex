%%%%%%%%%%%%%%%%%%%%%%%%%%%%%%%%%%%%%%%%%%%%%%%%%%%%%%%
%%%%%%%%%%%%%%%%% DISCUSSION %%%%%%%%%%%%%%%%%%%%%%%%
%%%%%%%%%%%%%%%%%%%%%%%%%%%%%%%%%%%%%%%%%%%%%%%%%%%%%%%

\chapter{Discussion}
\label{Discussion}

This experiment investigated the intrinsic geometry of the visual space by exocentric pointing in a virtual reality. It examined if the geometry is of constant curvature as proposed by Luneburg \citeyear{Luneburg.1947}. The aim was to both show similar results as Koenderink et al. \citeyear{Koenderink.2000} and to test if geodesics are indeed the best way to describe the curvature. The study looked at different factors possibly influencing this geometry such as the lighting, the angle in which the pointer is standing relatively to the targets and the subject ($\beta$), and the side on which the pointer is standing. Additionally, it was tested if virtual reality is a possible measure for examining the intrinsic geometry of the visual space. 

We have shown in our experiment that there was a main effect of $\beta$ for all subjects, a main effect for lighting, pointer position, and subject position for some subjects. Additionally to these main effects, some interactions have been found. Only for VP2 the interaction between pointer position and lighting was significant. Furthermore, an interaction between lighting and subject position was found for most subjects, an interaction between pointer position and subject position, and an interaction between $\beta$ and subject position was found for some subjects. 
Over all subjects %Formulierung
the reaction time was significantly shorter for the light condition.
Finally, a comparison of the data of VP5 showed similar deviations for our VR experiment as for the RL experiment. 

Hereby our main hypothesis, that the intrinsic visual geometry is of no constant curvature, can be supported by our data. Luneburg's theory of a constant hyperbolic curvature \citeyear{Luneburg.1947} seems to become more unlikely. But while our deviation generally shows dependence on the subject position, figure \ref{VP1Curves} shows particularly well how our results oppose preceding research and our hypothesis that the curvature is elliptic in near space and hyperbolic in far space.
The curvature of the visual space, when plotted depending on the distance of the subject to the target, is hyperbolic in near space and elliptic in far space in our experiment, which is directly the opposite to the results of Koenderink et al. \citeyear{Koenderink.2000}. They showed the visual space to be elliptic in near space and hyperbolic in far space. 

This contradiction could partly be justified by the different experimental conditions. Only the light condition can be compared to their experiment. The experiment of Koenderink et al. \citeyear{Koenderink.2000} took place in the field, hence under natural conditions just as in our light condition. There were still some differences between these experiments. Firstly, the tested distance between subject and targets varied only from 2.23m to 5.47m in our experiment in comparison to a few to over 20m in the experiment of Koenderink et al. \citeyear{Koenderink.2000}. Secondly, in our experiment the sight was very limited by walls and furniture, which all provided monocular depth cues. The effect of these should not be underestimated since in our experiment the dark and light condition showed significantly different deviations for some subjects. Overall this is still not enough to explain the contradictory results. 
But looking at the greater picture %Formulierung
it could be shown that the geometry of the intrinsic visual space has a curvature and that it is not constant. The possible influence of factors other than distance will be discussed in the following. 

The main effect of $\beta$ for the deviation cannot be explained easily. As $\beta$ directly depends on the subject position and the target, it is not clear if this main effect is indeed due to the angle, i.e. the perspective the subject is having on the target and the pointer, or due to one of its dependent factors. 

The curve of the deviation by $\beta$ shows a relative decrease of the angle deviation value in the beginning, a strong rise and a slight decrease at the highest values of $\beta$ (see e.g. figure \ref{DeviationVP1}). It must be said that this shape is particularly defined by the deviation values at very low and very high values of $\beta$, which are only supported by few data.
%fewer data or less data. only small data samples?
Leaving out these two extreme values, for some subjects one could only speak of a linear shape of the deviation. Especially the deviation values at $\beta = 0$ should be handled with care. The value $\beta = 0$ means that the pointer, the target, and the subject were in one line; the subject was directly pointing at themself when pointing at the target. Hence, this was rather an aiming task, which is much easier than exocentric pointing. This could explain the deviation being closer to $0$ for all subjects and small error bars. This gives thus the possibly incorrect impression of a decrease at lower values of $\beta$. Therefore, one should rather speak of a tendency of these decreases at the extreme values of $\beta$ and should put the general linear increase into focus.

If, on the other hand, the slight decrease of the deviation at high values of $\beta$ is the beginning of a strong decline, then our results could still be compatible with those of Koenderink et al. \citeyear{Koenderink.2000}. One could imagine that the visual space is hyperbolic in very near space, elliptic in near space, but again hyperbolic in far space. The data collected in this experiment is not sufficient to tell anything about this assumption. Further research should again focus on the type of curvature depending on the angle or the distance between subject and stimuli.

In general, the experimental setting allowed only small data sets to be comparable as many factors were varied such as the target and thus only few trials were made under the exact same conditions. This was due to the limiting size of the lab room. Using virtual reality will resolve this limitation for future experiments. In similar experimental settings one should make sure that all angles are supported by data of at least two subject positions. This could be achieved by varying the distance between pointer and target as it has been done in the experiment of Koenderink et al. \citeyear{Koenderink.2000}.

% ditch is no professional expression for sure
Another factor we tested was the influence of monocular depth cues on the curvature. For this purpose we changed between the light and the dark condition. The lighting was very significant for VP1 and VP3. This means, for these subjects the cues given by the room did influence how they perceived depth. Both had relative larger values in the dark condition. For VP1 the small relative decrease of the deviation seen at the smaller values of $\beta$ disappears in the light condition (see figure \ref{DeviationVP1}). In this condition the deviation tends to be positive for all values of $\beta$. An explanation for this cannot be given easily. But as the only difference in the two conditions is the additional information provided by the room, the room structure must be the reason for these more positive deviations. Therefore, in our experiment in the dark condition the visual space is mainly elliptic for some subjects, while in the light condition it is hyperbolic at small values of $\beta$ and elliptic at high values of $\beta$.

For VP3, on the other hand, the deviation is generally smaller in the light condition than in the dark condition (see figure \ref{DeviationVP3}). Taking into account that VP3 is a subject who also had similar reaction times in the light and the dark condition (see figure \ref{ReactionTime}), one could assume that this subject in contrast to the others makes better use of the monocular cues for determining the ideal pointer's orientation. In theory, there were enough cues given (such as seeing that the middle target is directly opposite of the pointer) so that one could approximate the ideal orientation very well. I propose that VP3 indeed tried this, which took them more time, but resulted in a more accurate pointer orientation, i.e. smaller deviations. With the same argument one can explain the reaction time and the deviation of VP4.

Another difference between the dark and the light condition can be seen when looking at the symmetry. The side  the pointer was standing at influenced the deviation for some subjects. Looking at this factor of symmetry we got results opposing our hypothesis. We expected the deviation to be symmetrical in the dark and possibly less symmetrical in the light condition. Our data shows directly the opposite. The pointer position (left or right) was significant for VP2 and VP3. But in both cases it was the dark condition in which they showed less symmetry, as an interaction between lighting and pointer position is given at least for VP2.

These results question the general assumption that the geodesic alone is enough to describe the curvature of the intrinsic visual space, because symmetry must be given if one wants to describe the curvature using geodesics. One must add that this is the first experiment showing such an asymmetry. In all preceding experiments symmetry was found and hence assumed. Therefore it is particularly important to wonder about what kind of factors could have made this asymmetry possible in our experiment.

Figure \ref{VP2Symmetry} can lead to the assumption that the symmetry depends on $\beta$, thus that fewer symmetry is found for greater values of $\beta$. As already said, as $\beta$ depends on the target and the subject position, no further exploration of this assumption can be made. The symmetry may thus also depend on the subject position as the interaction between pointer position and subject position for VP3 would suggest, but it cannot be said confidently. 

One possible explanation for the greater symmetry in the light condition is that the room provided more depth cues, hence more cues for the position of the target. On the other hand, the data suggests that in the dark condition the subjects were not able to determine a constant position of the targets. It seems that in this experiment binocular vision as the only depth cue is not enough to determine the position of one luminous light. Luneburg gives support for this thought, stating that experiments with an isolated point in the dark showed that "binocular observation of a single point does not differ from monocular observation" \cite[p.~629]{Luneburg.1950}. This is a major criticism of our experiment. Especially in virtual reality, in which the field of view is limited, this could have a great impact. Subjects had to turn their heads away from the pointer in order to see the target. In no trial both the target and the pointer were fully visible in one field of view. Hence the subjects were very often looking at isolated points in the dark and Luneburg's critique can be applied. 

Extending the remark on the limited field of sight, the small distance between stimuli and subject leads to many head movements. This means that an interaction such as for example between lighting and subject position may in fact be an interaction between lighting and head movements, caused by limited sight. To investigate this, the tracked head movements could be examined and be tested for a correlation to the angle deviation. 

At last, we expected our VR experiment to have similar results to the RL experiment. Our expectations were met: both experiments showed very similar results for VP5 (see figure \ref{DeviationBOTHVP5}). Especially in the light condition the deviations are extraordinarily similar. Therefore, one can claim that it is possible to measure the intrinsic geometry of the visual space by using virtual reality. Only in one condition the deviations scarcely overlap. The reason for this could be the order in which the trials were presented. VP5 started with the pointer left condition while being placed at subject position 2, i.e. the closest one to the pointer. This is the subject position in which most head movements have to be made in order to see the target and could thus be called the subject position with the most difficult trials. At this point in the VR experiment the subject has no clear conception of the room the trials are placed in, as these were the first trials. This idea is supported by figure \ref{DeviationVP2} as VP2, starting with the same condition as VP5, also shows asymmetry comparing the dark pointer left condition with the dark pointer right condition. 

In contrary, in RL the light condition was done before the dark condition and therefore the subjects were able to get an impression of the position of the targets and the room dimensions.
This may have led to these strongly differing deviations in the pointer left dark condition, whereas in the other conditions VP5 showed very similar results in both experiments. Of course one subject is not enough to make a final conclusion about the use of virtual reality. It should particularly taken with caution, because VP5 was a subject that was familiar to virtual reality, which is not given for every subject. In total it still allows the conclusion that virtual reality is a possible approach when measuring the intrinsic geometry of the visual space. 

% i havent explicitly talked about the interactions
In conclusion, this experiment gave evidence for a non-constant curvature of the intrinsic geometry of the visual space. According to our results, the visual space is hyperbolic in near space and elliptic in far space. As these are opposing results to preceding research, the curvature depending on the distance should be tested further. Our data showed that the deviation from the correct angle was influenced by the surrounding room structure. Virtual reality, which has indicated itself as a possible technique when measuring the intrinsic geometry of the visual space using exocentric pointing, will be of great help. In future research this influence of the depth cues given by the environment should be investigated by for example manipulating the position of the walls or placing objects close by the stimuli.
%macht der ganze ansatz noch sinn?
%es ist kein absoluter, festgesetzter Raum. das ist ziemlich widerlegt mit den hell dunkel unterschieden



% naive mit trennpunkten
% left/right vs left and right
% how many meters is the near space in Koenderink et al.